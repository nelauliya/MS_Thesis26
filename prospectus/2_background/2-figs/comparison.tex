\begin{comment} 
    Learning-based controllers have been known to achieve higher levels of performance with respect to more complex actions, typically at the cost of computational efficiency and/or formal guarantees of safety for the robot's behavior \cite{brunke_safe_2022}. Since the formulation of $\mathbf{g(x,u)}$ and $u_{\text{learned}}$ is handled by the learning medium, we are left without the ability to extract detailed functional information regarding either of them. 
\end{comment}
\begin{table}[h!]
    \centering
    \begin{tabularx}{\textwidth}{
        >{\hsize=0.2\hsize\centering\arraybackslash}X|
        >{\hsize=0.4\hsize\centering\arraybackslash}X|
        >{\hsize=0.4\hsize\centering\arraybackslash}X
        }
        Analysis        & Model-Based & Learning-Based\\ \hline\hline
        \raggedleft Advantages      & 
        Meaningful information regarding the dynamics and inputs of the system are preserved &
        Known to achieve higher maneuvering performance and better robustness\newline
        \cite{rakhmatillaev_integrative_2025}\\ \hline
        \raggedleft Disadvantages   &
        Less robust and adaptable at handling system configurations in fringe scenarios\newline
        \cite{rakhmatillaev_integrative_2025}, \cite{brunke_safe_2022} &
        More complex and typically lacks formal guarantees of safety\newline
        \cite{brunke_safe_2022}\\
    \end{tabularx}
    \caption{An assessment of Model- and Learning-Based controls.}
    \label{comparison}
\end{table}