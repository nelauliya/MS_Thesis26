%---------| background outline |---------%
%---------| 1. Define what model-based controls is
\subsection{Model-based Controls in Soft Robotics} \label{modelbasedinsoro}
Much of controls theory as a field has progressed in parallel to the advances in our understanding of dynamic modeling. In fact, much of the advancements in controls theory were driven by the need to supplement the gaps in our dynamic models. Starting from the frequency domain and linear controllers, to nonlinear controllers, and most recently machine learning. 

The implementation of control strategies for standard rigid robots has followed this line of progression. Algorithms built around the dynamical model of the robot itself came first, then only recently did strategies employing machine learning began to be implemented--mostly to handle unpredictable scenarios beyond the models' assumptions. Control strategies of the former kind are known as \textit{model}-based controls, where controllers are designed based on models that mathematically represent the robot's dynamics \cite{rakhmatillaev_integrative_2025}. It should be apparent that this has been the case since the dynamics of a rigid-bodied robot can more readily be modelled. 

On the other hand, the development of control strategies for \textit{soft} robots have followed the opposite direction: much of the early works in soft robotics controls employed machine learning strategies due to the high level of complexity and dimensionality required in the dynamical models. Thus, \textit{learning}-based controls were the model for soft robot control strategies through the formative years of the field. Only in recent years did this precedent began to be re-assessed, particularly with the advent of \textit{finite}-dimensional modelling (FDM) techniques compatible with the continuum dynamics of soft robots \cite{della_santina_model-based_2023}.
%----------------------------------------%     
%---------| 2. Model-based vs, learning-based; 
\subsection{Model- vs. Learning-based Controls} \label{modelvlearndef}
% \begin{figure}[h!]
    \tikzstyle{block} = [draw, rectangle, 
    minimum height=2em, minimum width=2em]
    \tikzstyle{sum} = [draw, circle]
    \tikzstyle{input} = [coordinate]
    \tikzstyle{output} = [coordinate]
    \tikzstyle{pinstyle} = [pin edge={to-,thin,black}]
    \centering
    \begin{tikzpicture}
        \node[input,name=into] {};
        \node[block, right=of into] (bmatrix) {$\mathbf{B}$};
        \node[sum, right=of bmatrix, inner sep = 0.6em
        ] (sum1) {};
        \node[block, isosceles triangle, right=of sum1
        ] (integ) {$\int$};
        \node[block, below=of integ, 
        ] (dynamic) {$\mathbf{A}$};
        \node[output, name=out, right= 4em of integ] {};

        \draw [->] (into) -- node[name=insig, pos=0.1, above] {$\mathbf{u}$} (bmatrix);
        \draw [->] (integ) -- node[name=outof, above] {$\mathbf{x}$} (out);
        \draw [->] (bmatrix) -- node[name=sum1a, pos=0.9, above] {$+$} (sum1);
        \draw [->] (dynamic) -| node[name=sum1b, pos=0.9, left] {$+$} (sum1);
        \draw [->] (sum1) -- node[name=insys, above] {$\mathbf{\dot{x}}$} (integ);
        \draw [->] (outof) |- (dynamic);
    \end{tikzpicture}
    \caption{A generic steady-state dynamic system, with an input 
    $\mathbf{u} \neq f(\mathbf{x})$
    making it open-loop.}
    \label{openloop}
\end{figure}
Model- and learning-based controls differ in the fundamental methodology that underpin \textit{how} each approach steers their dynamic systems. Consider the soft robot whose dynamics is represented as $\underline{\dot{q}}=\mathbf{f}(\underline{q},\underline{u})$, where $\underline{q} = [q_1,q_2,...,q_n]^T \in \mathbb{R}^n$ is the state parametrization used to describe the robot's configuration in space and $\underline{u}$ is some input signal. The application of a model- and learning-based controller, respectively, to the system may be represented according to \autoref{modelvslearn}.

\def\sysDyTerms{
    \underline{\dot{q}}=\mathbf{f}(\underline{q},\underline{u})
}

\begin{table}[!ht]
    \centering
    \begin{tabularx}{0.9\textwidth}{|*{2}{>{\centering\arraybackslash}X|}}
        \hline
        Model-based & Learning-based \\
        \hline
        \raggedright Given \par 
        \centering\arraybackslash $\begin{aligned}
            \sysDyTerms & \xrightarrow[\text{for}]{\text{solve}} \underline{u}(\underline{q},\underline{\bar{q}},...)\\
            \mathbf{f}(\underline{q},\underline{u}) & \rightarrow \mathbf{f}(\underline{q},\underline{u}(\underline{q},\underline{\bar{q}}))
        \end{aligned}$ \par
        \raggedright Such that \par 
        \centering\arraybackslash $\underline{\dot{q}} \approx \underline{\bar{\dot{q}}}$ \par & 
        \raggedright Given some dataset \par 
        \centering\arraybackslash 
            $X = \{(\underline{q_0},\underline{u_0}),\dots,(\underline{q_i},\underline{u_i})\}$ \par
        $\begin{aligned}
            X & \xrightarrow[\text{from}]{\text{extract}} \underline{u}^{\text{learned}}
        \end{aligned}$ \par
        \raggedright Such that \par
        \centering\arraybackslash
            $\underline{\dot{q}} \approx \underline{\bar{\dot{q}}}$ \\
        \hline
    \end{tabularx}
    \caption{High-level generalization of model- vs. learning-based controllers, where $\underline{\bar{\dot{q}}}$, $\underline{\bar{q}}$ describes the desired/reference configuration.}
    \label{modelvslearn}
\end{table}

% \raggedright\arraybackslash Where $\underline{\bar{\dot{q}}}$, $\underline{\bar{q}}$ describes the desired/reference configuration

\begin{comment}
    The distinction between model- and learning-based control approaches may be described using the functional framework presented above. In both approaches, some input $\mathbf{u}$ is now introduced to the previously steady-state system. The system becomes \textit{closed}-loop when a controller that uses the system output to determine the control input is implemented (see \autoref{openvsclosed}). %<-- need diagram to help explain!
    In this scenario, $f(\mathbf{x,e,t,...})$ can be considered the controller. It functionally relates the system output/current state--and typically error $\mathbf{e}$ and time as well--to the control input. 
\end{comment}

% \begin{figure}[h!]
    \centering
    \begin{tikzpicture}
        \bXInput[$\mathbf{r}$]{a}
        \bXComp*{b}{a}{}{-}{+}{}
        \bXLink{a}{b}
        \bXBlocL[3]{c}{
            $\mathbf{u}=f(...)$        
        }{b}
        \bXLinkName[0.7]{b-c}{$\mathbf{e}$}
        \bXLinkName[2.5]{c}{Controller}
        \bXBlocL[3]{d}{
            $\mathbf{\dot{x}=Ax + Bu}$
        }{c}
        \bXLinkName[0.7]{c-d}{$\mathbf{u}$}
        \bXLinkName[2.5]{d}{System}
        \bXOutput[4]{e}{d}
        \bXLink{d}{e}
        \bXLinkName[0.7]{d-e}{$\mathbf{x}$}
        \bXBranchy{d-e}{f}
        \bXBlocr[8]{g}{Feedback}{f}
        \bXLinkyx{d-e}{g}
        \bXLinkxy{g}{b}
    \end{tikzpicture}
    \caption{A generic closed-loop system, the control input is a function of desired state $\mathbf{r}$ and output $\mathbf{x}$.}
    \label{openvsclosed}
\end{figure}


For model-based controls, designing the controller based on the dynamics of the systems means solving for $\underline{u}$ such that it steers the system's output configuration towards the reference configuration. Using the terminology of the framework presented above, this inherently requires an expression for $\mathbf{f}$ to be known. The model-based approach as a whole revolves around tailoring $\mathbf{f}$ such that it lends itself for solving $\underline{u}$. 
  
Formulating models that allow the input to be solved in a mathematically manageable way, while simultaneously remaining faithfully representative of the \textit{actual} system behavior becomes one of model-based controls' biggest challenges. Learning-based controls circumvent this by using data-driven techniques and learning algorithms to arrive at $\mathbf{u}$. In some cases, $\mathbf{f}$ may even be left unknown as the full system is wholly formulated by the learning medium.

\begin{comment} 
    Learning-based controllers have been known to achieve higher levels of performance with respect to more complex actions, typically at the cost of computational efficiency and/or formal guarantees of safety for the robot's behavior \cite{brunke_safe_2022}. Since the formulation of $\mathbf{g(x,u)}$ and $u_{\text{learned}}$ is handled by the learning medium, we are left without the ability to extract detailed functional information regarding either of them. 
\end{comment}
\begin{table}[h!]
    \centering
    \begin{tabularx}{\textwidth}{
        >{\hsize=0.2\hsize\centering\arraybackslash}X|
        >{\hsize=0.4\hsize\centering\arraybackslash}X|
        >{\hsize=0.4\hsize\centering\arraybackslash}X
        }
        Analysis        & Model-Based & Learning-Based\\ \hline\hline
        \raggedleft Advantages      & 
        Meaningful information regarding the dynamics and inputs of the system are preserved &
        Known to achieve higher maneuvering performance and better robustness\newline
        \cite{rakhmatillaev_integrative_2025}\\ \hline
        \raggedleft Disadvantages   &
        Less robust and adaptable at handling system configurations in fringe scenarios\newline
        \cite{rakhmatillaev_integrative_2025}, \cite{brunke_safe_2022} &
        More complex and typically lacks formal guarantees of safety\newline
        \cite{brunke_safe_2022}\\
    \end{tabularx}
    \caption{An assessment of Model- and Learning-Based controls.}
    \label{comparison}
\end{table}

A general assessment of the benefits and drawbacks between the two approaches is presented in \autoref{comparison}. While both approaches bring different things to the table--and extensive research continue to be conducted on both, the author has decided to explore the model-based approach for this proposed thesis.
%----------------------------------------%